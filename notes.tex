\documentclass[10pt,a4paper]{article}
\usepackage{hyperref}
\usepackage{cleveref}
\hypersetup{hypertexnames = false, bookmarksdepth = 2, bookmarksopen = true, colorlinks, linkcolor = black, citecolor = black, urlcolor = black, pdfstartview={XYZ null null 1}}

\usepackage{amsfonts}
\usepackage[fleqn, leqno]{amsmath}
\usepackage{amsthm}
\usepackage{biblatex}
\usepackage{booktabs}
\usepackage{diagbox}
\usepackage{enumitem}
\usepackage{fixltx2e}
\usepackage{mathtools}
\usepackage{thmtools}
\usepackage{tikz-cd}
\usepackage[colorinlistoftodos]{todonotes}
\usepackage{xparse}
\usepackage{xspace}

\usepackage[T1]{fontenc}
\usepackage[charter]{mathdesign}
\usepackage[scaled]{beramono,berasans}
\usepackage{eucal}
\usepackage{epstopdf}
\usepackage{microtype}
\frenchspacing

\addbibresource{bibliography.bib}

\addtolength\parskip{.4ex}
\setlength\parindent{0cm}

\relpenalty=10000
\binoppenalty=10000

% todonotes configuration
\newcounter{todocounter}
\DeclareDocumentCommand\addreference{g}{\stepcounter{todocounter}\todo[color = blue!30, fancyline]{\thetodocounter. Add reference\IfNoValueF{#1}{: #1}}\xspace}
\DeclareDocumentCommand\checkthis{g}{\stepcounter{todocounter}\todo[color = red!50, fancyline]{\thetodocounter. Check this\IfNoValueF{#1}{: #1}}\xspace}
\DeclareDocumentCommand\fixthis{g}{\stepcounter{todocounter}\todo[color = orange!50, fancyline]{\thetodocounter. Fix this\IfNoValueF{#1}{: #1}}\xspace}
\DeclareDocumentCommand\expand{g}{\stepcounter{todocounter}\todo[color = green!50, fancyline]{\thetodocounter. Expand\IfNoValueF{#1}{: #1}}\xspace}
\newcommand\removethis{\stepcounter{todocounter}\todo[color=yellow!50]{\thetodocounter. Remove this?}}

% environments
\declaretheoremstyle[
  spaceabove = 3pt,
  spacebelow = 3pt,
]{lecture}
\theoremstyle{lecture}
\newtheorem{theorem}{Theorem}
\newtheorem{corollary}[theorem]{Corollary}
\newtheorem{definition}[theorem]{Definition}
\newtheorem{example}[theorem]{Example}
\newtheorem{lemma}[theorem]{Lemma}
\newtheorem{proposition}[theorem]{Proposition}
\newtheorem{remark}[theorem]{Remark}


\mathchardef\mhyphen="2D
\newcommand\dash{\nobreakdash-\hspace{0pt}}
\newcommand\Ab{\ensuremath{\mathrm{Ab}}}
\newcommand\bounded{\ensuremath{\mathrm{b}}}
\newcommand\Coh{\ensuremath{\mathrm{Coh}}}
\newcommand\coh{\ensuremath{\mathrm{coh}}}
\newcommand\dd{\mathrm{d}}
\newcommand\derived{\ensuremath{\mathbf{D}}}
\newcommand\fid{\ensuremath{\mathrm{fid}}}
\newcommand\Flat{\ensuremath{\mathrm{Flat}}}
\newcommand\identity{\ensuremath{\mathrm{id}}}
\newcommand\Inj{\ensuremath{\mathrm{Inj}}}
\newcommand\inj{\ensuremath{\mathrm{inj}}}
\newcommand\KKK{\ensuremath{\mathbf{K}}}
\newcommand\LLL{\ensuremath{\mathbf{L}}}
\newcommand\qc{\ensuremath{\mathrm{qc}}}
\newcommand\Qcoh{\ensuremath{\mathrm{Qcoh}}}
\newcommand\RR{\ensuremath{\mathrm{R}}}
\newcommand\RRR{\ensuremath{\mathbf{R}}}
\newcommand\yy{\ensuremath{\mathrm{y}}}
\newcommand\zz{\ensuremath{\mathrm{z}}}

\DeclareMathOperator\Ch{Ch}
\DeclareMathOperator\codim{codim}
\DeclareMathOperator\cousin{E}
\DeclareMathOperator\Div{Div}
\DeclareMathOperator\ddual{\underline{D}}
\DeclareMathOperator\dual{D}
\DeclareMathOperator\Ext{Ext}
\DeclareMathOperator\hh{h}
\DeclareMathOperator\HH{H}
\DeclareMathOperator\HHom{\mathcal{H}\mathit{om}}
\DeclareMathOperator\Hom{Hom}
\DeclareMathOperator\Pic{Pic}
\DeclareMathOperator\Proj{Proj}
\DeclareMathOperator\quotient{Q}
\DeclareMathOperator\res{res}
\DeclareMathOperator\RRRHom{\mathbf{R}Hom}
\DeclareMathOperator\RRRHHom{\mathbf{R}\mathcal{H}\mathit{om}}
\DeclareMathOperator\sheafExt{\mathcal{E}\mathit{xt}}
\DeclareMathOperator\Spec{Spec}
\DeclareMathOperator\supp{supp}
\DeclareMathOperator\tr{tr}
\DeclareMathOperator\Tr{Tr}


\title{Dimension functions: depth, measuring singularities}
\author{Pieter Belmans}
\date{February 14, 2014}

\begin{document}
\maketitle

\begin{abstract}
  These are the notes for my lecture on dimension functions in the ANAGRAMS seminar. The goal is to introduce the notion of \emph{depth}, and use this to study singularities. So the main subject are actually Gorenstein and Cohen--Macaulay rings.
\end{abstract}

\tableofcontents

\clearpage

\section{Definitions}
\label{section:definitions}
All rings will be commutative (strictly speaking this is not necessary, but we will take algebraic geometry as main motivation) and unital. Let us for the ease of statement assume that they are noetherian too (!). Often we will be working with local rings, but whenever we do so we will be explicit about it.

The main references are \cite{sga2,eisenbud-commutative-algebra,serre-algebre-locale,24-hours-of-local-cohomology}. Each of them has a different take on the subject, different emphasis and different goals. Whenever results aren't referenced probably, it is best to look in \cite{eisenbud-commutative-algebra}. For a more condensed approach to the subject, see \cite{24-hours-of-local-cohomology}, which is perfect for cursory reading.

We have seen \emph{Krull dimension} in a previous lecture, which was a measure of ``how big the ring is'', geometrically speaking: the Krull dimension of the ring~$A$ is equal to the dimension\footnote{It is a noetherian topological space, hence dimension is defined as the supremum over the lengths of descending chains of closed subsets. So they must agree, by the very definition of the Zariski topology.} of the topological space~$\Spec A$.

But Krull dimension alone is not a sufficient measure in algebraic geometry. If we consider singularities we want to know some properties of them: can we ``measure'' how singular things are? Of course, the dimension of the singular locus is something that could interest us: if we intersect two lines in a single point, or we intersect two planes in a line, we get a measure on the size of the singular locus. But what if we want to know something about ``how complicated'' a singularity is?

\subsection{Regular sequences}
Something that we can study besides Krull dimension is the \emph{codimension} of an ideal~$I$ in~$A$. The question becomes: how does~$\Spec A/I$ relate to~$\Spec A$? As we have taken~$A$ noetherian we have that~$I$ is finitely generated. Is there a way of choosing the generators of~$I$ in such a way that we can deduce something interesting?

Remark that codimension is of course defined regardless of how we generate~$I$. But considering the generators as hypersurfaces we wish to compute codimension by counting the number of times we have interested hypersurfaces. By Krull's principal ideal theorem intersecting with a hypersurface let's the dimension drop with at most one.
\begin{definition}
  Let~$A$ be a ring. Let~$M$ be a module over~$A$. A sequence of elements~$x_1,\dotsc,x_n$ in~$A$ is a \emph{regular sequence} on~$M$ if
  \begin{enumerate}
    \item $(x_1,\dotsc,x_n)M\neq M$;
    \item $x_i$ is a non-zerodivisor in~$M/(x_1,\dotsc,x_{i-1})M$ for all~$i=1,\dotsc,n$.
  \end{enumerate}
\end{definition}
\begin{example}
  Take~$A=k[x,y,z]$ with~$k$ a field. Take~$M=A$. Then the most classical regular sequence (if we wish to describe a point) would be
  \begin{equation}
    x,y,z.
  \end{equation}
  This one is a bit boring. Remark that we could also consider shorter regular sequences if one is interested in higher-dimensional results. Now let's look at:
  \begin{equation}
    x,y(1-x),z(1-x),
  \end{equation}
  which is a regular sequence:
  \begin{enumerate}
    \item we have~$1\notin(x,y(1-x),z(1-x))=(x,y,z)$;
    \item $x$ is not a zerodivisor in~$k[x,y,z]$, $y$ is not a zerodivisor in~$k[y,z]$ and~$z$ is not a zerodivisor in~$k[z]$.
  \end{enumerate}
  On the other hand,
  \begin{equation}
    y(1-x),z(1-x),x
  \end{equation}
  is not a regular sequence:
  \begin{enumerate}
    \item the ideal generated is the same as for the previous regular sequence, so the first condition is still satisfied;
    \item the condition fails at the second step:
      \begin{equation}
        z(1-x)y=zy-zxy=zy-zy=0,
      \end{equation}
      so~$z(1-x)$ is a zerodivisor in~$A/(x_1)A$, which is not allowed.
  \end{enumerate}
  Hence the order of the regular sequence is important\footnote{One can prove though that if an ideal is generated by some regular sequence, then we can find generators for this ideal such that they form a regular sequence for any permutation \cite[exercise 17.6]{eisenbud-commutative-algebra}. I haven't done this exercise, but one sees that taking the regular sequence~$x,y,z$ suffices in this case. The reader is invited to do the general exercise and tell me about it.}.
\end{example}
Fortunately, if we look at local rings, things are nicer \cite[corollary 17.2]{eisenbud-commutative-algebra}. And as we intend to study singularities this is good enough.
\begin{lemma}
  Let~$(A,\mathfrak{m})$ be a local ring\footnote{Recall that all rings are noetherian. I won't repeat this from now on, but it's important to know this}. If~$a_1,\dotsc,a_n$ is a regular sequence with~$a_i\in\mathfrak{m}$, then any permutation is again a regular sequence.
\end{lemma}
One now sees some analogy with the notion of a regular local ring.
\begin{definition}
  Let~$(A,\mathfrak{m})$ be a local ring. If~$\mathfrak{m}=(a_1,\dotsc,a_n)$ is a way of generating~$\mathfrak{m}$ with a minimal number of generators such that~$\dim A=n$, then~$A$ is \emph{regular}.
\end{definition}
So for every regular local ring of Krull dimension~$n$ we get a regular sequence of length~$n$. The goal of this note is to show that by relaxing the definition of regular we can get many results for singular rings which are analogous to the regular case.

\subsection{Depth}
\label{subsection:depth}
Knowing what a regular sequence is we can introduce depth. Or give a seemingly totally unrelated definition.
\begin{definition}
  Let~$A$ be a ring. Let~$I$ be an ideal of~$A$. Let~$M$ be a finitely generated module over~$A$ such that~$IM\neq M$. Then the \emph{depth} of~$I$ with respect to~$M$ is the number
  \begin{equation}
    \depth_I(M)\coloneqq\min_{i\in\mathbb{N}}\left\{ i\mid\Ext_A^i(A/I,M)\neq 0 \right\}
  \end{equation}
\end{definition}
Unfortunately, this definition does not have a geometric flavour to it. It is known as \emph{codimension homologique} in \cite{serre-algebre-locale}, for which the motivation (besides the obvious homological flavour to the definition) can be found in \cite[proposition 21]{serre-algebre-locale}. The term \emph{grade} is also in use by the way.

The following result by David Rees shows that we do have a geometric interpretation.
\begin{theorem}
  Let~$(A,\mathfrak{m})$ be a local ring. Let~$M$ be a finitely generated~$A$\dash module. Then all maximal regular sequences~$x_1,\dotsc,x_n$ for~$M$ such that~$x_i\in\mathfrak{m}$ have length equal to~$\depth_{\mathfrak{m}}(M)$.
\end{theorem}
So the depth is a measure of how much we can cut down things using hypersurfaces, where each intersection has to drop the dimension by~1. This also yields the bound
\begin{equation}
  \label{equation:depth-dim-inequality}
  \depth_{\mathfrak{m}}(A)\leq\dim A
\end{equation}
as cutting down by a hypersurface can at most drop the dimension by~1 (recall Krull's principal ideal theorem). So the depth of a ring (i.e.\ the depth of the ring over itself) measures how much we can cut things down in a non-trivial way.

The terminology is of course related to the height of a prime ideal: height is dimension, depth ``tries to be'' codimension.

\begin{example}
  Let~$A=k[x,y]/(x^2,xy)$. This represents the affine line with an embedded double point at the origin. So we localise this ring with respect to the origin. The Krull dimension of this ring is~$1$: it is the local ring of the affine line (plus some lower-dimensional stuff).
  
  The depth of this local ring is~$0$: we cannot cut down the embedded component in the direction of the~$y$\dash axis any further. If there was no ``fuzzy direction'', i.e.\ if we started with~$k[x,y]/(xy)$ there wouldn't be a problem. We would have two one-dimensional components, related to the two axes, and we can cut them down by a hypersurface. So the problem is related to the ring having components of different dimension, which is a condition that has to be satisfied for the equality \eqref{equation:depth-dim-inequality} to hold, as seen in \cref{subsection:counterexamples}.
\end{example}

For more examples of computing the depth of a ring we refer to \cref{section:properties}, where we give some more geometric examples of the (in)equality \eqref{equation:depth-dim-inequality}.


\subsection{Cohen--Macaulay rings}
The (in)equality \eqref{equation:depth-dim-inequality} allows to make the following definition.
\begin{definition}
  Let~$(A,\mathfrak{m})$ be a local ring. Then~$A$ is \emph{Cohen--Macaulay} if
  \begin{equation}
    \depth_{\mathfrak{m}}(A)=\dim A.
  \end{equation}
  Let~$A$ be any ring. Then~$A$ is \emph{Cohen--Macaulay} if the localisation~$A_{\mathfrak{p}}$ is Cohen--Macaulay for all~$\mathfrak{p}\in\Spec A$.
\end{definition}
As a matter of fact, it suffices to impose this condition only at the closed points \cite[proposition 18.8]{eisenbud-commutative-algebra}.

How can we interpret this? In some sense we can say that Krull dimension measures things from bottom to top: we start from the zero-dimensional parts and go up. Depth on the other hand is characterised by Rees in terms of regular sequences: these cut the whole ring down using hypersurfaces. So Cohen--Macaulay rings are those rings for which it doesn't matter in which way we measure things: depth and height are complementary.

\subsection{Gorenstein rings}
To define the notion of a Gorenstein ring we will take the ``easier'' or ``modern'' definition that is used nowadays. In \cref{lemma:gorenstein-CM-relation} the equivalence with a more classical or geometric definition will be discussed.

In the previous lecture we have seen homological dimensions for rings. If we consider a ring as a module over itself, its projective dimension is rather boring (a free module of rank one is a projective resolution of itself). In the case of injective dimension we get something interesting: a ring is rarely injective over itself. If the ring is reduced, being self-injective is equivalent to being~$0$\dash dimensional (in the Krull dimension sense). So the next best thing to ask for is:
\begin{definition}
  Let~$A$ be a local ring. Then~$A$ is \emph{Gorenstein} if~$A$ as an~$A$\dash module has finite injective dimension.
  
  Let~$A$ be a ring. Then~$A$ is \emph{Gorenstein} if the localisation~$A_{\mathfrak{p}}$ is Gorenstein for every~$\mathfrak{p}\in\Spec A$.
\end{definition}

We haven't argumented why having finite injective dimension is a reasonable thing to ask. That will be done when discussing the properties shortly. An interesting thing to remark is that Serre has characterised regular local rings as exactly those rings for which the global dimension is finite (and equal to the Krull dimension).

\begin{remark}
  Gorenstein rings are named after Daniel Gorenstein, who introduced them in 1952\ in the case of (singular) points on curves. This zero-dimensional case was generalised later on by Bass \cite{bass-ubiquity-gorenstein-rings}, Grothendieck \cite{grothendieck-theoremes-de-dualite} (studying duality results, as discussed in a previous lecture series!) and Serre. Famously, Daniel Gorenstein used to say that he didn't understand the definition of a Gorenstein ring himself. He is mostly known for his great contributions to the classification of finite simple groups.
\end{remark}

The wonderful article \cite{bass-ubiquity-gorenstein-rings} shows the equivalence of the homological definition to a more geometric notion, related to the Cohen--Macaulayness we just introduced. Let's quote (parts of) this result.
\begin{theorem}
  \label{lemma:gorenstein-CM-relation}
  Let~$A$ be a ring. The following conditions are equivalent:
  \begin{enumerate}
    \item $A$ is Gorenstein (so here we defined it to have finite injective dimension at all local rings);
    \item $A_{\mathfrak{p}}$ is Cohen--Macaulay for every~$\mathfrak{p}\in\Spec A$ (respectively~$\mathfrak{p}\in\MaxSpec A$) and some system of parameters generates an irreducible ideal in~$A_{\mathfrak{p}}$;
    \item $A_{\mathfrak{p}}$ is Cohen--Macaulay for every~$\mathfrak{p}\in\Spec A$ (respectively~$\mathfrak{p}\in\MaxSpec A$) and every system of parameters generates an irreducible ideal in~$A_{\mathfrak{p}}$.
  \end{enumerate}
\end{theorem}
Recall that irreducible ideals are ideals which cannot be written as the intersection of two larger ideals. Every prime ideal is irreducible, every irreducible ideal is primary. The closed subset defined by an irreducible ideal is irreducible as a topological space.



\section{Properties}
\label{section:properties}
\subsection{Relations between regular, Gorenstein and Cohen--Macaulay rings}
\label{subsection:relations}
So we have three notions of rings, each of them motivated by measuring how singular things are. How do these relate? The answer is really nice: we have the sequence
\begin{enumerate}
  \item regular;
  \item Gorenstein;
  \item Cohen--Macaulay;
\end{enumerate}
where each condition is (strictly) more general than the one above. That a regular ring is Cohen--Macaulay is easy to see from the properties discussed in \cref{section:definitions}, this was proved by Macaulay for polynomial rings and by Cohen for formal power series. This motivates the name Cohen--Macaulay.

The fact that regular rings are Gorenstein is covered by the discussion on complete intersections in \cref{subsection:complete-intersections}. That Gorenstein rings are Cohen--Macaulay follows from \cref{lemma:gorenstein-CM-relation}.

Examples showing that each condition is strictly more general than the one above are given in \cref{subsection:counterexamples}.

\subsection{Complete intersections}
\label{subsection:complete-intersections}
We now discuss an important class of Gorenstein rings.
\begin{definition}
  Let~$V$ be an algebraic variety inside~$\mathbb{P}_k^n$ such that~$\dim V=m$. Then~$V$ is a \emph{complete intersection} if the ideal describing~$V$ can be generated by exactly~$n-m$ elements (and no more).
\end{definition}
This means that~$V$ has the ``right'' codimension: each equation in the ideal describing~$V$ defines a hypersurface, the dimension can drop by at most~$1$. Hence we require that the hypersurfaces intersect eachother in such a way that the maximal codimension is achieved. Remark that the resulting variety can have a large singular locus, but that is okay as long as the codimension of the whole thing is as expected.
\begin{example}
  Examples of complete intersections are abound: any hypersurface will do.
\end{example}
\begin{example}
  The easiest example of a variety which is \emph{not} a complete intersection is the twisted cubic. It is the curve in~$\mathbb{P}_k^3$ given as the image of
  \begin{equation}
    \mathbb{P}_k^1\to\mathbb{P}_k^3:[s:t]\mapsto[s^3:s^2t:st^2:t^3].
  \end{equation}
  Hence locally on an affine chart it is~$(t,t^2,t^3)$ (up to some reordering). This curve is described by the (homogeneous) ideal
  \begin{equation}
    (xz-y^2,yw-z^2,xw-yz)
  \end{equation}
  in the graded ring~$k[x,y,z,w]$. So we need three (quadratic) equations, but we end up with a codimension~$1$ and not a codimension~$0$ subvariety.

  Set-theoretically speaking everything is nice: it is the intersection of the quadric surface~$xz-y^2=0$ and the cubic surface~$z(yw-z^2)-w(xw-yz)=0$. But ideal-theoretically speaking the twisted cubic should have degree~$3$ by a generalisation of the B\'ezout theorem, which is only possible if we intersect a plane with a cubic surface. This is not possible, as no four distinct points on the twisted cubic are coplanar, but if we intersect with a plane everything is coplanar.
\end{example}
Introducing the notion of being a complete intersection is interesting because we have \cite[corollary 21.19]{eisenbud-commutative-algebra} (translating things to local rings instead of projective spaces):
\begin{theorem}
  \label{theorem:complete-intersections-are-Gorenstein}
  Let~$A$ be a regular local ring. If~$I$ is an ideal generated by a regular sequence (i.e.\ $A/I$ is a complete intersection) then~$A/I$ is Gorenstein.
\end{theorem}
The proof uses the fact that we have a Koszul resolution by the regular sequence, which is a minimal free resolution of~$A/I$. This gives us information on the highest Ext-group of~$A/I$ and~$A$, and by the relationship between dualising objects and Gorenstein rings \cite[theorem 21.15]{eisenbud-commutative-algebra} we get the result.

\subsection{Counterexamples}
\label{subsection:counterexamples}
So we have a hierarchy of being singular:
\begin{enumerate}
  \item regular;
  \item Gorenstein;
  \item Cohen--Macaulay.
\end{enumerate}
Moreover we have seen that each condition implies the one above it. Now we come to the counterexamples!

\paragraph{Gorenstein but not regular}
First some rings that are Gorenstein, but not regular.
\begin{example}
  As every complete intersection is Gorenstein by \cref{theorem:complete-intersections-are-Gorenstein} it suffices to take a complete intersection which is not regular. Just taking a hypersurface that is singular suffices:~$k[x,y]/(x^2-y^3)$. This is a one-dimensional ring, with a mild singularity at the origin (so it is the local ring at the origin which is Gorenstein but not regular, all the others are of course regular).
\end{example}
\begin{example}
  Or we could take~$k[x]/(x^2)$, which is a non-reduced point.
\end{example}
\begin{example}
  Or we could take~$k[x,y]/(x^2,y^3)$, again fuzzy, but now in two different directions. So it's a complete intersection, but not regular for two different reasons.
\end{example}
It is also interesting to have (non-regular) Gorenstein rings which are not complete intersections. To obtain one we have to do some effort, the following results severely limit the places where we can find examples:
\begin{enumerate}
  \item almost complete intersections (i.e.\ if the codimension of~$V$ is~$m$ then the defining ideal has a minimal set of generators of size~$m+1$) are never Gorenstein;
  \item in codimension 2 we have that being Gorenstein is equivalent to being a complete intersection \cite[corollary 21.20]{eisenbud-commutative-algebra};
  \item \ldots
\end{enumerate}
So to write down a possible example, we should at least start from the ring~$k[x,y,z]$ (or its local version~$k[[x,y,z]]$) for codimension~$3$ to make sense, and we will need to have (at least) five defining equations (four would be an almost complete intersection, but oddly enough these are never Gorenstein).
\begin{example}
  Consider
  \begin{equation}
    k[x,y,z]/(x^2,y^2,xz,yz,z^2-xy),
  \end{equation}
  which is a bunch of fuzzy things in 3 different directions in the origin.
\end{example}
Now that we have seen a counterexample, we can discuss some other restrictions that we know, if one wishes to find more complicated examples:
\begin{enumerate}
  \item if the codimension~$c$ is~3 (e.g.\ as in the previous example) then the number of generators must be odd, so we cannot find an example with six generators;
  \item if the codimension~$c$ is~$\geq 4$ we can find examples for any number of generators~$\geq c+2$, but some other restrictions apply.
\end{enumerate}

\paragraph{Cohen--Macaulay but not Gorenstein}
The restriction that we can use now is:
\begin{enumerate}
  \item in codimension 1 we have that Cohen--Macaulay is equivalent to Gorenstein \cite[corollary 21.20]{eisenbud-commutative-algebra};
\end{enumerate}
So we want something that is not a complete intersection (because these are Gorenstein), but rather an almost complete intersection (which are never Gorenstein). Moreover we want it to be in codimension~2.
\begin{example}
  Consider
  \begin{equation}
    k[x,y]/(x^2,y^2,xy)
  \end{equation}
  which is again a bunch of fuzzy stuff around the origin.
\end{example}

\paragraph{Not Cohen--Macaulay}
To find a ring that is not Cohen--Macaulay we can do several things. The following results tell us where to look:
\begin{enumerate}
  \item Cohen--Macaulay rings are universally catenary \cite[corollary 18.10]{eisenbud-commutative-algebra};
  \item Cohen--Macaulay rings are equidimensional \cite[corollary 18.11]{eisenbud-commutative-algebra};
  \item Hartshorne's connectedness principle: if~$A$ is Cohen--Macaulay, $I$ and~$J$ are proper ideals of~$A$ such that their radicals are incomparable then necessarily~$\codim(I+J)\leq 1$, \cite[theorem 18.12]{eisenbud-commutative-algebra}.
\end{enumerate}
This first condition is virtually useless: rings that are not universally catenary are really hard to come up with. For an example, see \cite[tag 02JE]{stacks}. So, it is possible to find examples of rings that are not Cohen--Macaulay this way, but it's not really enlightening.

The second condition is more promising. Recall that being \emph{equidimensional} means that all the maximal ideals have the same codimension, and all minimal primes have the same dimension. For a local ring (without embedded components) this reduces to the geometric property of all irreducible components having the same dimension. So finding an example becomes easy.
\begin{example}
  The intersection of a plane and a line in a single point is not Cohen--Macaulay. At the singular point there are two minimal primes: one corresponding to the plane and one corresponding to the line. These don't have the same dimension, hence the ring is not Cohen--Macaulay.
\end{example}
The example given in \cref{subsection:depth} also falls into this class of counterexamples.

The third condition is also interesting, and is related to what we will discuss in \cref{subsection:intersection-multiplicity}. Geometrically speaking we have to find a situation in which we have two things intersection eachother in a single point such that removing the point from the spectrum of the local ring makes the resulting topological space disconnected. 
\begin{example}
  Consider the intersection of two planes inside~$\mathbb{A}_k^4$, in a single point. This is possible (think linear algebra) and has a picture associated to it. If we remove the closed point in the local ring for the intersection (which is of codimension 2) we end up with two components which are disjoint, hence this ring cannot be Cohen--Macaulay.
\end{example}

\subsection{Some facts}
\label{subsections:facts}
Without further ado, some random facts:
\begin{enumerate}
  \item $A$ local is Cohen--Macaulay if and only if~$\hat{A}$ is Cohen--Macaulay \cite[proposition 18.8]{eisenbud-commutative-algebra};
  \item $A$ local is Gorenstein if and only if~$\hat{A}$ is Gorenstein \cite[proposition 21.18]{eisenbud-commutative-algebra};
  \item $A$ is Cohen--Macaulay if and only if~$A[x]$ is Cohen--Macaulay \cite[proposition 18.9]{eisenbud-commutative-algebra};
  \item determinantal rings are Cohen--Macaulay \cite[theorem 18.18]{eisenbud-commutative-algebra};
  \item if~$G$ is a linearly reductive algebraic group acting by linear transformations on~$k[x_1,\dotsc,x_n]$ then the ring of invariants~$S^G$ is Cohen--Macaulay \cite[\S 18.5]{eisenbud-commutative-algebra};
\end{enumerate}

\subsection{Auslander--Buchsbaum formula}
\label{subsection:auslander-buchsbaum}
A less random fact is the \emph{Auslander--Buchsbaum formula} \cite[theorem 19.9]{eisenbud-commutative-algebra}. It relates the projective dimension to the depth by saying that they are complementary to eachother.
\begin{theorem}
  Let~$(A,\mathfrak{m})$ be a local ring. Let~$M$ be a finitely generatd~$A$\dash module such that~$\projdim_A(M)<+\infty$. Then
  \begin{equation}
    \projdim_A(M)+\depth_{\mathfrak{m}}(M)=\depth_{\mathfrak{m}}(A).
  \end{equation}
\end{theorem}
This allows one to obtain a recognition theorem for Cohen--Macaulay rings \cite[corollary 19.10]{eisenbud-commutative-algebra}.
\begin{corollary}
  Let~$(A,\mathfrak{m})$ be a local ring. If there exists some finitely generated~$A$\dash module of projective dimension equal to~$\dim A$ then~$A$ is Cohen--Macaulay.
\end{corollary}

\subsection{Intersection multiplicity}
\label{subsection:intersection-multiplicity}
Cohen--Macaulay rings are the type of rings for which intersection multiplicity ``works as expected''. The same phenomenon of ``working as expected'' is pervasive for these mildly singular rings, another manifestation will be seen in the next paragraph.

Due to lack of time and space (these notes are meant for a 30 minute lecture\ldots), I will just say that to compute intersection numbers in general one has to use Serre's Tor-formula, which (without any further explanation) reads
\begin{equation}
  \mu(X;Y,Z)=\sum_{i=0}^{+\infty}(-1)^i\length_{\mathcal{O}_{X,x}}\left( \Tor_i^{\mathcal{O}_{X,x}}(\mathcal{O}_{X,x}/\mathcal{I}_x,\mathcal{O}_{X,x}/\mathcal{J}_x) \right)
\end{equation}
and that for Cohen--Macaulay rings it suffices to consider the first term only. One is referred to \cite{serre-algebre-locale} for the whole story. The example of two planes in 4-space intersecting in a single point is an example of where it is possible to explicitly compute the terms in the Tor-formula, and see why we actually need it.


\subsection{Grothendieck duality}
During my lectures on Grothendieck duality I have often used the term ``mildly singular''. Depending on the context one has to think Gorenstein or Cohen--Macaulay when confronted with ``mildly singular''. Already in \cite{bass-ubiquity-gorenstein-rings} the link with Grothendieck duality is realised:
\begin{quote}
  The material of this paper seems to have connections with Grothendieck's Bourbaki expos\'e on duality theorems \cite{grothendieck-theoremes-de-dualite}, but which I am not competent to elaborate. Undoubtedly all of what follows is peripheral to the ideas of that paper.
\end{quote}
Recall that Grothendieck duality was concerned with the existence of a dualising complex \cite{hartshorne-residues-and-duality}, generalising the idea of duality of (finite-dimensional) vectorspaces. One of the main problems (besides its existence) is determining the properties of this dualising complex. If one knows something about how singular the scheme in question is we can get the answer, which is given in \cref{table:comparison-X-dualising-sheaf}.
\begin{table}[ht]
  \centering
  \begin{tabular}{cc}
    \toprule
    how nice is $X$? & how nice is $\omega_X^\circ$? \\\midrule
    $X$ smooth & $\omega_X^\circ=\bigwedge^{\dim X}\Omega_X[\dim X]$ \\
    $X$ Gorenstein & $\omega_X^\circ$ shift of a line bundle by $\dim X$ \\
    $X$ Cohen--Macaulay & $\omega_X^\circ$ shift of a sheaf by $\dim X$ \\
    $X$ arbitrary & $\omega_X^\circ$ is a complex \\
    \bottomrule
  \end{tabular}
  \caption{Comparison of the singularities of $X$ and the look of $\omega_X^\circ$}
  \label{table:comparison-X-dualising-sheaf}
\end{table}
So to summarise:
\begin{enumerate}
  \item if $X$ is Gorenstein the dualising complex is as nice as the smooth case;
  \item if $X$ is Cohen--Macaulay we could build up the theory without derived categories to some extent, as is done in \cite{hartshorne-algebraic-geometry}.
\end{enumerate}
The rather unintuitive algebraic notions (from a geometric or topological point of view) of being Gorenstein or Cohen--Macaulay therefore have important implications in algebraic geometry! When applied to affine schemes, it means that the dualising object for a Gorenstein ring is the ring itself (which is just like the situation for a regular ring), whereas for a Cohen--Macaulay ring it will be ``just'' a module.

\printbibliography

\end{document}

\documentclass[10pt,a4paper]{article}
\usepackage{hyperref}
\usepackage{cleveref}
\hypersetup{hypertexnames = false, bookmarksdepth = 2, bookmarksopen = true, colorlinks, linkcolor = black, citecolor = black, urlcolor = black, pdfstartview={XYZ null null 1}}

\usepackage{amsfonts}
\usepackage[fleqn, leqno]{amsmath}
\usepackage{amsthm}
\usepackage{biblatex}
\usepackage{booktabs}
\usepackage{diagbox}
\usepackage{enumitem}
\usepackage{fixltx2e}
\usepackage{mathtools}
\usepackage{thmtools}
\usepackage{tikz-cd}
\usepackage[colorinlistoftodos]{todonotes}
\usepackage{xparse}
\usepackage{xspace}

\usepackage[T1]{fontenc}
\usepackage[charter]{mathdesign}
\usepackage[scaled]{beramono,berasans}
\usepackage{eucal}
\usepackage{epstopdf}
\usepackage{microtype}
\frenchspacing

\addbibresource{bibliography.bib}

\addtolength\parskip{.4ex}
\setlength\parindent{0cm}

\relpenalty=10000
\binoppenalty=10000

% todonotes configuration
\newcounter{todocounter}
\DeclareDocumentCommand\addreference{g}{\stepcounter{todocounter}\todo[color = blue!30, fancyline]{\thetodocounter. Add reference\IfNoValueF{#1}{: #1}}\xspace}
\DeclareDocumentCommand\checkthis{g}{\stepcounter{todocounter}\todo[color = red!50, fancyline]{\thetodocounter. Check this\IfNoValueF{#1}{: #1}}\xspace}
\DeclareDocumentCommand\fixthis{g}{\stepcounter{todocounter}\todo[color = orange!50, fancyline]{\thetodocounter. Fix this\IfNoValueF{#1}{: #1}}\xspace}
\DeclareDocumentCommand\expand{g}{\stepcounter{todocounter}\todo[color = green!50, fancyline]{\thetodocounter. Expand\IfNoValueF{#1}{: #1}}\xspace}
\newcommand\removethis{\stepcounter{todocounter}\todo[color=yellow!50]{\thetodocounter. Remove this?}}

% environments
\declaretheoremstyle[
  spaceabove = 3pt,
  spacebelow = 3pt,
]{lecture}
\theoremstyle{lecture}
\newtheorem{theorem}{Theorem}
\newtheorem{corollary}[theorem]{Corollary}
\newtheorem{definition}[theorem]{Definition}
\newtheorem{example}[theorem]{Example}
\newtheorem{lemma}[theorem]{Lemma}
\newtheorem{proposition}[theorem]{Proposition}
\newtheorem{remark}[theorem]{Remark}


\mathchardef\mhyphen="2D
\newcommand\dash{\nobreakdash-\hspace{0pt}}
\newcommand\Ab{\ensuremath{\mathrm{Ab}}}
\newcommand\bounded{\ensuremath{\mathrm{b}}}
\newcommand\Coh{\ensuremath{\mathrm{Coh}}}
\newcommand\coh{\ensuremath{\mathrm{coh}}}
\newcommand\dd{\mathrm{d}}
\newcommand\derived{\ensuremath{\mathbf{D}}}
\newcommand\fid{\ensuremath{\mathrm{fid}}}
\newcommand\Flat{\ensuremath{\mathrm{Flat}}}
\newcommand\identity{\ensuremath{\mathrm{id}}}
\newcommand\Inj{\ensuremath{\mathrm{Inj}}}
\newcommand\inj{\ensuremath{\mathrm{inj}}}
\newcommand\KKK{\ensuremath{\mathbf{K}}}
\newcommand\LLL{\ensuremath{\mathbf{L}}}
\newcommand\qc{\ensuremath{\mathrm{qc}}}
\newcommand\Qcoh{\ensuremath{\mathrm{Qcoh}}}
\newcommand\RR{\ensuremath{\mathrm{R}}}
\newcommand\RRR{\ensuremath{\mathbf{R}}}
\newcommand\yy{\ensuremath{\mathrm{y}}}
\newcommand\zz{\ensuremath{\mathrm{z}}}

\DeclareMathOperator\Ch{Ch}
\DeclareMathOperator\codim{codim}
\DeclareMathOperator\cousin{E}
\DeclareMathOperator\Div{Div}
\DeclareMathOperator\ddual{\underline{D}}
\DeclareMathOperator\dual{D}
\DeclareMathOperator\Ext{Ext}
\DeclareMathOperator\hh{h}
\DeclareMathOperator\HH{H}
\DeclareMathOperator\HHom{\mathcal{H}\mathit{om}}
\DeclareMathOperator\Hom{Hom}
\DeclareMathOperator\Pic{Pic}
\DeclareMathOperator\Proj{Proj}
\DeclareMathOperator\quotient{Q}
\DeclareMathOperator\res{res}
\DeclareMathOperator\RRRHom{\mathbf{R}Hom}
\DeclareMathOperator\RRRHHom{\mathbf{R}\mathcal{H}\mathit{om}}
\DeclareMathOperator\sheafExt{\mathcal{E}\mathit{xt}}
\DeclareMathOperator\Spec{Spec}
\DeclareMathOperator\supp{supp}
\DeclareMathOperator\tr{tr}
\DeclareMathOperator\Tr{Tr}


\title{Dimension functions: depth, measuring singularities}
\author{Pieter Belmans}
\date{February 14, 2014}

\begin{document}
\maketitle

\begin{abstract}
  These are the notes for my lecture on dimension functions in the ANAGRAMS seminar. The goal is to introduce the notion of \emph{depth}, and use this to study singularities.
\end{abstract}

\tableofcontents

\clearpage

\section{Definitions}
Before we start: let's try to motivate things a little. All rings will be commutative (strictly speaking not necessary, but we'll take algebraic geometry as main motivation) and unital. Let's for the ease of statement assume that they are noetherian too (!). Often we will be working with local rings, but whenever we do so we will be explicit about it.

We have seen \emph{Krull dimension} in a previous lecture, which was a measure of ``how big the ring is'', geometrically speaking: the Krull dimension of the ring~$A$ is equal to the dimension of the topological space~$\Spec A$\footnote{This is a noetherian topological space, hence dimension is defined as the supremum over the lengths of descending chains of closed subsets. So they must agree, by the very definition of the Zariski topology.}.

\subsection{Depth and regular sequences}
Another thing that we can study is the \emph{codimension} of an ideal~$I$ in~$A$: how does~$\Spec A/I$ relate to~$\Spec A$? As we have taken~$A$ noetherian we have that~$I$ is finitely generated. Is there a way of choosing the generators of~$I$ in such a way that we can deduce something interesting?
\begin{definition}
  Let~$A$ be a ring. Let~$M$ be an~$A$\dash module. A sequence of elements~$x_1,\dotsc,x_n\in A$ is a \emph{regular sequence} on~$M$ if
  \begin{enumerate}
    \item $(x_1,\dotsc,x_n)M\neq M$;
    \item $x_i$ is a non-zerodivisor on~$M/(x_1,\dotsc,x_{i-1})M$ for all~$i=1,\dotsc,n$.
  \end{enumerate}
\end{definition}
\begin{example}
  Take~$A=k[x,y,z]$ with~$k$ a field. Take~$M=A$. Then the most classical regular sequence would be
  \begin{equation}
    x,y,z.
  \end{equation}
  But this one is boring. Let's look at:
  \begin{equation}
    x,y(1-x),z(1-x),
  \end{equation}
  which is is a regular sequence:
  \begin{enumerate}
    \item we have~$1\notin(x,y(1-x),z(1-x))$;
    \item $x$ is not a zerodivisor in~$k[x,y,z]$, $y$ is not a zerodivisor in~$k[y,z]$ and~$z$ is not a zerodivisor in~$k[z]$.
  \end{enumerate}
  On the other hand,
  \begin{equation}
    y(1-x),z(1-x),x
  \end{equation}
  is not a regular sequence:
  \begin{enumerate}
    \item the ideal generated is the same as for the previous regular sequence, so the first condition is still satisfied;
    \item the condition fails already at the second step:
      \begin{equation}
        z(1-x)y=zy-zxy=zy-zy=0.
      \end{equation}
  \end{enumerate}
  Hence the order of the regular sequence is important\footnote{One can prove though that if an ideal is generated by some regular sequence, then we can find generators for this ideal such that they form a regular sequence for any permutation \cite[exercise 17.6]{eisenbud-commutative-algebra}. I haven't done this exercise, nor tried to come up with generators for the ideal under consideration that work under any permutation. The reader is invited to do this and tell me about it.}.
\end{example}
Fortunately, if we look at local rings, things are nicer \cite[corollary 17.2]{eisenbud-commutative-algebra}.
\begin{corollary}
  Let~$A$ be a local ring\footnote{Recall that all rings are noetherian. I won't repeat this from now on, but it's important to know this} with maximal ideal~$\mathfrak{m}$. If~$a_1,\dotsc,a_n$ is a regular sequence with~$a_i\in\mathfrak{m}$, then any permutation is again a regular sequence.
\end{corollary}
One now sees some analogy with the notion of a regular local ring.
\begin{definition}
  Let~$A$ be a local ring with maximal ideal~$\mathfrak{m}$. If~$\mathfrak{m}=(a_1,\dotsc,a_n)$ is a way of generating~$\mathfrak{m}$ with a minimal number of generators such that~$\dim A=n$, then~$A$ is \emph{regular}.
\end{definition}
So for every regular local ring of Krull dimension~$n$ we get a regular sequence of length~$n$.

\subsection{Cohen--Macaulay rings}

\subsection{Gorenstein rings}
To define the notion of a Gorenstein ring we will take the ``easier'' or ``modern'' definition that is used nowadays. In \cref{lemma:gorenstein-CM-relation} the equivalence with the classical definition will be discussed.

In the previous lecture we have seen homological dimensions for rings. If we consider a ring as a module over itself, its projective dimension is rather boring (a free module of rank one is a projective resolution of itself). In the case of injective dimension we get something interesting: a ring is rarely injective over itself. If the ring is reduced, being self-injective is equivalent to being~$0$\dash dimensional (in the Krull dimension sense).

\begin{definition}
  Let~$A$ be a local ring. Then~$A$ is \emph{Gorenstein} if~$A$ as an~$A$\dash module has finite injective dimension.
  
  Let~$A$ be a ring. Then~$A$ is \emph{Gorenstein} if the localisation~$A_{\mathfrak{p}}$ is Gorenstein for every~$\mathfrak{p}\in\Spec A$.
\end{definition}

We haven't argumented why having finite injective dimension is a reasonable thing to ask. That will be done when discussing the properties shortly.

\begin{remark}
  Gorenstein rings are named after Daniel Gorenstein, who introduced them in 1952 \cite{gorenstein-arithmetic-theory-adjoint-plane-curves} in the case of singular points curves. This zero-dimensional case was generalised later on by Bass \cite{bass-ubiquity-gorenstein-rings}, Grothendieck \cite{grothendieck-theoremes-de-dualite} (studying duality results, as discussed in a previous lecture series!) and Serre \cite{serre-sur-les-modules-projectifs}. Famously, Daniel Gorenstein used to say that he didn't understand the definition of a Gorenstein ring himself. He is mostly known for his great contributions to the classification of finite simple groups.
\end{remark}


\section{Properties}
\subsection{Relations between regular, Gorenstein and Cohen--Macaulay rings}

\subsection{Complete intersections}
We now discuss an important class of Gorenstein rings.
\begin{definition}
  Let~$V$ be an algebraic variety inside~$\mathbb{P}_k^n$ such that~$\dim V=m$. Then~$V$ is a \emph{complete intersection} if the ideal describing~$V$ can be generated by exactly~$n-m$ elements (and no more).
\end{definition}
This means that~$V$ has the ``right'' codimension: each equation in the ideal describing~$V$ defines a hypersurface, the dimension can drop by at most~$1$. Hence we require that the hypersurfaces intersect eachother in such a way that the maximal codimension is achieved.
\begin{example}
  % some positive examples
\end{example}
\begin{example}
  The easiest example of a variety which is \emph{not} a complete intersection is the twisted cubic. It is the curve in~$\mathbb{P}_k^3$ given as the image of
  \begin{equation}
    \mathbb{P}_k^1\to\mathbb{P}_k^3:[s:t]\mapsto[s^3:s^2t:st^2:t^3].
  \end{equation}
  Hence locally on an affine chart it is~$(t,t^2,t^3)$ (up to some reordering). This curve is described by the (homogeneous) ideal
  \begin{equation}
    (xz-y^2,yw-z^2,xw-yz)
  \end{equation}
  in the graded ring~$k[x,y,z,w]$. So we need three (quadratic) equations, but we end up with a codimension~$1$ and not a codimension~$0$ subvariety.

  Set-theoretically speaking everything is nice: it is the intersection of the quadric surface~$xz-y^2=0$ and the cubic surface~$z(yw-z^2)-w(xw-yz)=0$. But ideal-theoretically speaking the twisted cubic should have degree~$3$ by a generalisation of the B\'ezout theorem, which is only possible if we intersect a plane with a cubic surface. This is not possible, as no four distinct points on the twisted cubic are coplanar, but if we intersect with a plane everything is coplanar.
\end{example}
Introducing this notion is interesting, because we have \cite[corollary 21.19]{eisenbud-commutative-algebra} (translating things to local rings instead of projective spaces):
\begin{theorem}
  \label{theorem:complete-intersections-are-Gorenstein}
  Let~$A$ be a regular local ring. If~$I$ is an ideal generated by a regular sequence (i.e.\ $A/I$ is a complete intersection) then~$A/I$ is Gorenstein.
\end{theorem}
The proof uses the fact that we have a Koszul resolution by the regular sequence, which is a minimal free resolution of~$A/I$. This gives us information on the highest Ext-group of~$A/I$ and~$A$, and by the relationship between dualising objects and Gorenstein rings \cite[theorem 21.15]{eisenbud-commutative-algebra} we get the result.

\subsection{Some other facts}

\subsection{Counterexamples}
So we have a hierarchy of being singular:
\begin{enumerate}
  \item regular (not singular at all);
  \item Gorenstein;
  \item Cohen--Macaulay.
\end{enumerate}
Moreover we have seen that each condition implies the one above it. Now we come to the counterexamples!

\paragraph{Gorenstein but not regular}
First some rings that are Gorenstein, but not regular.
\begin{example}
  As every complete intersection is Gorenstein by \cref{theorem:complete-intersections-are-gorenstein} it suffices to take a complete intersection which is not regular. Just taking a hypersurface that is singular suffices:~$k[x,y]/(x^2-y^3)$. This is a one-dimensional ring, with a mild singularity at the origin (so it is the local ring at the origin which is Gorenstein but not regular, all the others are of course regular).
  
  Or we could take~$k[x]/(x^2)$, which is a non-reduced point.
\end{example}
It is also interesting to have (non-regular) Gorenstein rings which are not complete intersections. To obtain one we have to do some effort, the following results severely limit the places where we can find examples:
\begin{enumerate}
  \item almost complete intersections (i.e.\ if the codimension of~$V$ is~$m$ then the defining ideal has a minimal set of generators of size~$m+1$) are never Gorenstein \addreference;
  \item in codimension 2 we have that being Gorenstein is equivalent to being a complete intersection \cite[corollary 21.20]{eisenbud-commutative-algebra};
  \item \ldots
\end{enumerate}
So to write down a possible example, we are (at least) required to start from~$k[x,y,z]$ (or its local version~$k[[x,y,z]]$) for codimension~$3$ to make sense, and we will need to have (at least) five defining equations (four would be an almost complete intersection, but oddly enough these are never Gorenstein).
\begin{example}
  
\end{example}
Now that we have seen a counterexample, we can discuss some other restrictions that we know, if one wishes to find more complicated examples:
\begin{enumerate}
  \item if the codimension~$c$ is~3 (e.g.\ as in the previous example) then the number of generators must be odd \cite{buchsbaum-eisenbud}, so we cannot find an example with six generators;
  \item if the codimension~$c$ is~$\geq 4$ we can find examples for any number of generators~$\geq c+2$, but some other restrictions apply.
\end{enumerate}

\paragraph{Cohen--Macaulay but not Gorenstein}

\begin{enumerate}
  \item in codimension 1 we have that Cohen--Macaulay is equivalent to Gorenstein \cite[corollary 21.20]{eisenbud-commutative-algebra};
\end{enumerate}



\begin{enumerate}
  \item CM: Hartshorne's connectedness theorem (non-example: planes in P4)
  \item CM: equidimensional
  \item CM: preservation properties (polynomial rings, localisation, completion)
  \item Gorenstein: codimension versus complete intersection
  \item dimension arguments for Gorenstein (see 21.3)
  \item complete intersections are Gorenstein
  \item codimension 1: Gorenstein = CM = the ideal I for A=R/I with R regular is principal
  \item codimension 2: I has to be generated by a regular sequence
\end{enumerate}

\printbibliography

\end{document}

\documentclass[10pt,a4paper]{article}
\input{packages}
\input{configuration}
\input{macros}

\title{Dimension functions: depth, measuring singularities}
\author{Pieter Belmans}
\date{February 14, 2014}

\begin{document}
\maketitle

\begin{abstract}
  These are the notes for my lecture on dimension functions in the ANAGRAMS seminar. The goal is to introduce the notion of \emph{depth}, and use this to study singularities.
\end{abstract}

\tableofcontents

\clearpage

\begin{enumerate}
  \item CM: Hartshorne's connectedness theorem (non-example: planes in P4)
  \item CM: equidimensional
  \item CM: preservation properties (polynomial rings, localisation, completion)
  \item Gorenstein: codimension versus complete intersection
  \item dimension arguments for Gorenstein (see 21.3)
  \item complete intersections are Gorenstein
  \item codimension 1: Gorenstein = CM = the ideal I for A=R/I with R regular is principal
  \item codimension 2: I has to be generated by a regular sequence
\end{enumerate}

\printbibliography

\end{document}
